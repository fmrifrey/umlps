% Source: http://tex.stackexchange.com/a/150903/23931
\documentclass[11pt]{article}
\usepackage[letterpaper,margin=1in]{geometry}
\usepackage{xcolor}
\usepackage{fancyhdr}
\usepackage{hyperref}
\renewcommand{\headrulewidth}{1.5pt}
%\usepackage{tgschola} % or any other font package you like
\usepackage{times}
\usepackage{titlesec}
\titlespacing*{\section}
{0pt}{0.5ex}{0.7ex}
%\renewcommand{\familydefault}{\rmdefault}
\usepackage{lastpage}
\usepackage{color}
\usepackage{float}
\usepackage{graphicx}
\usepackage{amssymb} % For \odot operator
\usepackage{amsmath} % For equation* environment

\pagestyle{fancy}
\fancyhf{}
\fancyhead[C]{%
  \footnotesize
  \fontsize{12}{12}
  \large \soptitle\hspace{1cm}
  \institution \hspace{1.2cm}
  \yourname
  }
\fancyfoot[C]{\thepage}

\newcommand{\institution}{Coil Sketching}
\newcommand{\soptitle}{Update step Lipschitz constant calculation}
\newcommand{\yourname}{David Frey}
%\newcommand{\yourweb}{https://www.abcd.com/}
%\newcommand{\youremail}{email@address.edu}

\newcommand{\statement}[1]{\par\medskip
  \underline{\textcolor{blue}{\textbf{#1:}}}\space
}

\begin{document}

\noindent Let $H_S^{(t)} := (S^{(t)}A)^TS^{(t)}A$, and $G^{(t)} := A^T(Ax^{(t)}-b)$.
The IHS update step is computed as:
\begin{equation*}
  \begin{aligned}
    x^{(t+1)} &= \text{arg}\min_x \frac{1}{2}\|S^{(t)}A(x^{(t)}-x)\|_2^2 + (Ax^{(t)}-b)^TAx \\
    &= \text{arg}\min_x \frac{1}{2}(x^{(t)}-x)^T H_S^{(t)} (x^{(t)}-x) + (G^{(t)})^Tx \\
    &= \text{arg}\min_x \frac{1}{2}x^TH_S^{(t)}x - x^TH_S^{(t)}x^{(t)} + \frac{1}{2}(x^{(t)})^TH_S^{(t)}x^{(t)} + (G^{(t)})^T x \\
    &= \text{arg}\min_x \frac{1}{2}\|S^{(t)}Ax\|_2^2 - x^TH_S^{(t)}x^{(t)} + (G^{(t)})^T x \\
    &= \text{arg}\min_x \frac{1}{2}\|S^{(t)}Ax\|_2^2 - (H_S^{(t)}x^{(t)} - G^{(t)})^T x \\
  \end{aligned}
\end{equation*}
Let $\tilde{f}_S^{(t)}(x) := \frac{1}{2}\|SAx\|_2^2 - (H_Sx^{(t)} - G^{(t)})^T x$. Then the gradient can be calculated as:
\begin{equation*}
  \begin{aligned}
    \nabla \tilde{f}_S^{(t)}(x) &= (S^{(t)}A)^TS^{(t)}Ax - (H_S^{(t)}x^{(t)} - G^{(t)}) \\
    &= H_S^{(t)}x - (H_S^{(t)}x^{(t)} - G^{(t)}) \\
  \end{aligned}
\end{equation*}
Then:
\begin{equation*}
  \begin{aligned}
    \|\nabla \tilde{f}_S^{(t)}(x_1) -  \nabla \tilde{f}_S^{(t)}(x_1)\|_2
    &= \|H_S^{(t)}x_1 - H_S^{(t)}x_2 \|_2 \\
    &= \|H_S^{(t)}(x_1 - x_2)\|_2 \\
    &\leq \|H_S^{(t)}\|_2\|(x_1 - x_2)\|_2
  \end{aligned}
\end{equation*}
Therefore, $\nabla \tilde{f}_S^{(t)}(x)$ is L-Lipschitz with $L = \|H_S^{(t)}\|_2 = \|S^{(t)}A\|_2^2$.
Alternatively, by setting the gradient to zero at minimizer $x_*^{(t+1)}$, we get:
\begin{equation*}
  \begin{aligned}
    \nabla \tilde{f}_S^{(t)}(x_*^{(t+1)}) &= H_S^{(t)}x_*^{(t+1)} - (H_S^{(t)}x^{(t)} - G^{(t)}) &= 0 \\
    \rightarrow x_{**}^{(t+1)} &= \text{arg}\min_x \|H_S^{(t)}x - (H_S^{(t)}x^{(t)} - G^{(t)})\|_2^2
  \end{aligned}
\end{equation*}
Therefore, the IHS update step is equivalent to solving the following least squares problem:
\begin{equation*}
  x^{(t+1)} = \text{arg}\min_x \frac{1}{2}\|Bx - c\|_2^2
\end{equation*}
Where $B = H_S^{(t)}$ and $c = H_S^{(t)}x^{(t)} - G^{(t)}$.
This least squares problem is easier to solve and is $L$-Lipschitz with $L = \|B\|_2^2 = \|H_S^{(t)}\|_2^2 = \|S^{(t)}A\|_2^4$.

\end{document}